% dbht.tex
\documentclass[11pt]{article}
\usepackage[utf8]{inputenc}
\usepackage{graphicx} % support the \includegraphics command and optio
\usepackage[fleqn]{amsmath}
\usepackage{amssymb}
\usepackage{amsfonts}

\title{Double Barrier Hitting Time}
\author{Keith A. Lewis}

\begin{document}
\maketitle

\section{Notation}

Let \((B_t)_{t\ge0}\) be standard Brownian motion and define
\[
T = \inf\{t\ge0 : B_t > a\  {\rm or}\  B_t < -b\}
\] where \(a\) and \(b\) are positive. 
%Define \(T_a = \inf\{t \ge0 0 : B_t > a\}\)
%and \(T_b = \inf\{t \ge0 0 : B_t < -b\}\). 


\section{Approximation}
For \(\sigma\sqrt{t}\) is small, lognormal and normal models have
similar statistical properties 
\(e^{-\sigma^2t/2 + \sigma B_t}\approx (1 + \sigma)B_t\).
They both have the same expected value and the variance
\(e^{\sigma^2t} - 1 \approx \sigma^2 t\). Taking stock
price \(S_t = s(1 + \sigma B_t)\) define \(T_\epsilon\)
by
\[
T_\epsilon = \inf\{t : 
\max_{0\le s\le t} s(1 + \sigma B_s) > s(1 + \epsilon)\ {\rm or}\ 
\min_{0\le s\le t} s(1 + \sigma B_s) < s(1 - \epsilon)\}.
\]
We have
\begin{align*}
P(T < t) 
&= P(\max B_t > \epsilon/\sigma\ 
{\rm or}\ 
\min B_t < -\epsilon/\sigma)\\
&= P(E_\epsilon) + P(E_{-\epsilon}) - P(E_\epsilon\cap E_{-\epsilon})\\
\end{align*}
where \(E_a\) is the event \(\max B > a/\sigma\) and \(E_{-b}\) is
the event \(\min B < -b/\sigma\).

When \(\sigma\sqrt{t}\) is small we have
\(P(E_a\cap E_{-b}) \approx 0\) so
\begin{align*}
P(T < t) &\approx P(\max B > a) + P(\min B < -b)\\
	&= 2P(B > a) + P(\max B > b)\\
	&= 2(P(B > a) + P(B > b))\\
\end{align*}

If \(a = b = \epsilon/\sigma\) then 
\(P(T < t) = 4P(B_t > a) = 4(1 - N(\epsilon/\sigma\sqrt{t}))\)
where \(N\) is the standard normal distribution. The inverse
is \(t = (\epsilon/(\sigma N^{-1}(1 - p/4))^2\). 
To get a proper cdf  we should use
\(P(T < t) = \min\{4P(B_t > a), 1\}\) so \(T\) is
bounded by \((N^{-1}(3/4)\epsilon/\sigma)^2\). For
\(\epsilon = 0.01\) and \(\sigma = 0.2\) this works out
to about a quarter of a day.

%\section{Symmetric Barriers}
%Using the fact that if
%\[
%u(t,x) = \sum_{k=0}^n (-1/2)^k p^{(2k)}(x) t^k/k!
%\]
%for any polynomial \(p\), then \(u(t, B_t)\) is a martingale
%we can calculate the moments of \(T\) by taking \(p(x) = x^{2n}\).
%


%For \(n = 1\) we have \(u(t,x) = x^2 + (-1/2)2t = x^2 - t\).
%Using the optional stopping theorem
%\(0 = a^2 - ET\) so \(ET = a^2\).
%
%For \(n = 2\) we have 
%\begin{align*}
%u(t,x) &= x^4 + (-1/2)4\cdot 3x^2 t + (1/4)4\cdot 3\cdot 2 t^2/2\\
%	&= x^4 - 6x^2t + 3t^2
%\end{align*}
%and \(0 = a^4 - 6a^2ET + 3ET^2\) so \(ET^2 = (5/3)a^4\).
%
%For \(n = 3\) we have
%\begin{align*}
%u(t,x) &= x^6 + (-1/2)6\cdot 5x^4t + (1/4)6\cdot 5 \cdot 4\cdot 3x^2t^2 + (-1/8)6!t^3/3!\\
%&= x^6 - 15x^4t + 30x^2t^2 - 15t^3
%\end{align*}
%and \(0 = a^6 - 15a^4ET + 30a^2ET^2 - 15ET^3\) so
%\(0 = a^6 - 15a^6 + 50a^6 - 15ET^3\), hence
%\(ET^3 = (12/5)a^6\).
%
%For \(n = 4\) we have
%\begin{align*}
%u(t,x) &= x^8 \\
%&+ (-1/2)8\cdot 7x^6t \\
%&+ (1/4)8\cdot 7 \cdot 6\cdot 5x^4t^2 \\
%&+ (-1/8)(8!/2!)x^2t^3/3!\\
%&+ (1/16)8!t^4/4!\\
%&= x^8 -28a^6t + 84x^4t^2 - 420x^2t^3 + 105t^4\\
%\end{align*}
%and \(0 = a^8(1 - 28 + 84(5/2) - 420(12/5)) + 105ET^4\)
%so \(ET^4 = (55/7)a^8\).
%
%If \(p(x) = x^{2n}\) then \(p^{(2k)}(x) = (2n)!/(2(n-k))! x^{2(n-k)}\) so
%\[
%u_k(t,x) = \sum_{k=0}^n (-1/2)^k (2n)!/(2(n-k))! x^{2(n-k)}t^k/k!
%\]
%and
%\[
%0 = \sum_{k=0}^n (-1/2)^k (2n)!/(2(n-k))! a^{2(n-k)}E[T^k]/k!
%\]
%It is simple to show \(ET^k = \tau_ka^{2k}\) for constant \(\tau_k\) hence
%\[
%0 = a^{2n}\sum_{k=0}^n (-1/2)^k (2n)!/(2(n-k))! \tau_k/k!
%\]
%so
%\begin{align*}
%\tau_n &= (-2)^n n! \sum_{k=0}^{n-1} (-1/2)^k (2n)!/(2(n-k))! \tau_k/k!\\
%	&= \sum_{k=0}^{n-1} (-2)^{n-k} \frac{(2n)!}{(2(n-k))!}\frac{n!}{k!} \tau_k\\
%\end{align*}
%



\end{document}
