% jr.tex
\documentclass[11pt]{article}
\usepackage[utf8]{inputenc}
\usepackage{graphicx} % support the \includegraphics command and optio
\usepackage[fleqn]{amsmath}
\usepackage{amssymb}
\usepackage{amsfonts}

\newcommand{\Var}{\mathop{\rm{Var}}}
\newcommand{\Cov}{\mathop{\rm{Cov}}}

% \usepackage[parfill]{parskip} % Activate to begin paragraphs with an empty line rather than an indent

\title{Generalized Jarrow-Rudd Approximate Option Valuation}
\author{Keith A. Lewis}
\date{\today}

\begin{document}
\maketitle

\section{Outline}

Instead of working with the difference of the cumulants from a lognormal
distribution as in Jarrow-Rudd \cite{?} we work with the difference of
the cumulants of a normal distribution. 
This is more natural since the cumulants of a normal distribution are
easier to interpret.

In section 2 we recall some basic fact about cumulants.
Section 3 uses an Edgeworth expansion to give an explicit formula for
the cumulative distribution function of the perturbed distribution.
Option valuation is considered in section 4. We show how the
Esscher transform can be calculated using the same method as in
section 3. The K-model is described in section 5 as a parameterization
that seems to fit market data well. The last section collects some
general remarks.

\section{Cumulants}

The {\em cumulants}, \((\kappa_n)\), of a random variable \(X\)
are defined by
\[
\log Ee^{uX} = \sum_{n=0}^\infty \kappa_n \frac{u^n}{n!}
\]
Taking \(u = 0\) yields \(\kappa_0 = 0\). Since
\((d/du)^n\log Ee^{uX}|_{u = 0} = \kappa_n\) it is easy to
work out that
\(\kappa_1 = EX\) and \(\kappa_2 = \Var(X)\). Higher order
cumulants are less intuitive.

Cumulants have some handy properties. 
The cumulants of a random variable plus a constant are the 
same except the first cumulant is increased by the constant.
More generally, the cumulants of the sum of two independent 
random variables are the sums of the cumulants.
The \(n\)-th cumulant is homogeneous of degree \(n\), 
\(\kappa_n(cX) = c^n\kappa_n(X)\).

Another nice property of cumulants is that they are more likely
to exists than moments, \(m_n = EX^n\).
The relationship between cumulants and moments involves Bell polynomials\cite{?}.
\[
Ee^{uX} = \sum_{n=0}^\infty m_n \frac{u^n}{n!}
 = \exp(\sum_{n=1}^\infty \kappa_n \frac{u^n}{n!})
= \sum_{n=0}^\infty B_n(\kappa_1,\dots,\kappa_n) \frac{u^n}{n!}
\]
where \(B_n(\kappa_1,\dots,\kappa_n)\) is the \(n\)-th complete
Bell polynomial.
This is just a special case of the
Fa\`a di Bruno formula first proved by Louis Fran\c{c}ois Antoine
Arborgast in 1800\cite{Arborgast}.
Bell polynomials satisfy the recurrence \cite{Comtet} \(B_0 = 1\) and
\[
B_{n+1}(x_1,\dots,x_{n+1}) = \sum_{k=0}^n \binom{n}{k}
B_{n - k}(x_1,\dots, x_{n - k}) x_{k+1}.
\]

\subsection{Examples}
If \(X\) is
normal then \(Ee^X = e^{E X + \Var(X)/2}\) so
\(Ee^{uX} = e^{uEX + u^2\Var(X)/2}\) showing the
third and higher order cumulants vanish, if \(X\) is Poisson
with mean \(\mu\) then \(\kappa_n = \mu\) for all \(n\), and
if \(X\) is exponential with mean \(\mu\) then
\(\kappa_n = (n - 1)!\mu^n\).
%\(\chi^2\) with \(r\) degrees of freedom \(\kappa_n = 2^{n-1}(n-1)!r\).

\section{Edgeworth Expansion}
Given random
variables \(X\) and \(Y\), we have
\[\log E e^{iuY} - \log E e^{iuX} = \sum_{n=1}^\infty \Delta\kappa_n (iu)^n/n!\]
where \((\Delta\kappa_n)\) are the differences of the cumulants 
of \(X\) and \(Y\), so
\[
Ee^{iuY} = Ee^{iuX}\sum_{n=0}^\infty \frac{B_n}{n!}(iu)^n.
\]

Let \(F\) and \(G\) be the cumulative distribution functions of
\(X\) and \(Y\) respectively.
The simple fact that the Fourier transform of \(F'\) is \(-iu \hat F(u)\) 
implies the Fourier transform of the \(n\)-th derivative
\(F^{(n)}\) is \((-iu)^n\hat F(u)\).
Taking the inverse Fourier transform shows
\[
G(x) = \sum_{n=0}^\infty (-1)^n \frac{B_n}{n!} F^{(n)}(x)
\]

\subsection{Hermite Polynomials}
The derivatives the standard normal cumulative distribution 
can be computed using Hermite polynomials\cite{?}.
%Recall the Hermite polynomials are the result of the Gram-Schmidt
%process applied to the basis \((x^n)_{n\ge0}\) on the Hilbert space
%having inner product \((f,g) = \int_{-\infty}^\infty f(x)g(x)\,e^{-x^2/2}dx\).
The standard result\cite{?} is
\[
H_n(x) = (-1)^n e^{x^2/2}\frac{d^n}{dx^n}e^{-x^2/2}
\]
and they satisfy the recurence \(H_0(x) = 1\), \(H_1(x) = x\) and
\[
H_{n+1}(x) = xH_n(x) - n H_{n-1}(x).
\]
Note some authors use \(He_n(x)\) instead of \(H_n(x)\).

%We have
%\(
%F^{(1)}(x) = F'(x) = e^{-x^2/2}/\sqrt{2\pi} 
%= H_0(x) e^{-x^2/2}/\sqrt{2\pi}.
%\)
%More generally, 
%\[
%F^{(n)}(x) = (-1)^{n-1} H_{n-1}(x) e^{-x^2/2}/\sqrt{2\pi}.
%\]
%
%For appropriate \(\phi\), e.g., exponentially bounded, 
%integration by parts shows 
%\(\int_{-\infty}^\infty \phi(x)\,dF^{(n)}
%= - \int_{-\infty}^\infty \phi'(x)F^{(n)}\,dx
%= - \int_{-\infty}^\infty \phi'(x)\,dF^{(n-1)}\)
%so for any \(k\le n\)
%\[
%\int_{-\infty}^\infty \phi(x)\,dF^{(n)}
%= (-1)^k \int_{-\infty}^\infty \phi^{(k)}(x)\,dF^{(n-k)}.
%\]

%\subsection{Put Valuation}
%The forward value of a put option with strike \(k\) is
%\(E\max\{k - S, 0\}\) where 
%\(S = s\exp(-\kappa(t,\sigma) + \sigma\sqrt{t}Y)\)
%and \(\kappa(t\sigma) = \log E \exp(\sigma\sqrt{t}Y)\).
%so \(ES = s\).
%
%Let \(\phi(y) = \max\{k - se\exp(-\kappa(t,\sigma) + \sigma\sqrt{t}y)\)

\section{Option Valuation}
The forward value of a put on a stock \(S\) with strike \(k\) is
\[
E\max\{k - S,0\} = k P(S \le k) - E S\,P^*(S \le k)
\]
where \(dP^*/dP = S/ES\) is the {\em Esscher transform}.
As we will see, this is the analog of 
\(E\max\{k - S,0\} = kN(-d_2) - sN(-d_1)\)
when \(S\) is lognormal.

\subsection{Black Model}
The Black model is \(S = S_t = se^{-\sigma^2t/2 + \sigma B_t}\)
where \((B_t)\) is standard Brownian motion and \(t\) is time
in years. Note \(ES = s\) and \(S \le k\)
if and only if \(B_t/\sqrt{t} \le z\) where
\(z = (\log k/s + \sigma^2 t/2)/\sigma\sqrt{t}\). In the usual
notation \(z = -d_2\).

Letting \(X = B_t/\sqrt{t}\) and so \(F = N\) is the standard
normal cumulative distributing function we have
\(P(X\le z) = F(z)\) and \(P^*(X\le z) = P(X + \sigma\sqrt{t} \le z)
= F(z - \sigma\sqrt{t})\) where we use the fact
\[
Ee^U f(V) = Ee^U Ef(V + \Cov(U,V))
\]
if \(U\) and \(V\) are
jointly normal. In the standard notation \(z - \sigma\sqrt{t} = -d_1\).

\subsection{Generalized Black Model}
The generalized Black model is \(S = se^{-\kappa(t,\sigma) + \sigma C_t}\)
where \((C_t)\) is any stochastic process and 
\(\kappa(t, \sigma) = \log E \exp(\sigma C_t)\). 
Note \(ES = s\) and \(S \le k\)
if and only if 
\(C_t/\sqrt{t} \le (\log k/s + \kappa(t,\sigma))/\sigma\sqrt{t}\).

Letting \(Y = C_t/\sqrt{t}\) and \(G\) be the corresponding cumulative
distribution function, \(G(z) = P(Y\le z)\), we need to compute
\[
G^*(z) = P^*(Y\le z) = Ee^{-\kappa(t,\sigma) + \sigma\sqrt{t}Y}1(Y\le z).
\]
Letting \(\gamma = \sigma\sqrt{t}\), the cumulants of \(G^*\) can be found from
\begin{align*}
\log E e^{uY^*} &= \log E e^{-\kappa(t,\gamma) + \gamma Y} e^{uY}\\
&= -\kappa(t, \gamma) + \log E e^{(\gamma + u) Y}\\
&= -\sum_{n=1}^\infty \kappa_n \frac{\gamma^n}{n!} 
	+ \kappa_n \frac{(u + \gamma)^n}{n!}\\
&= \sum_{n=1}^\infty \sum_{k=0}^{n-1} \kappa_n \binom{n}{k}
	\frac{\gamma^{n - k}u^k}{n!}\\
&= \sum_{k=1}^\infty \sum_{n=k + 1}^\infty \kappa_n 
	\binom{n}{k}\frac{\gamma^{n - k}u^k}{n!}\\
&= \sum_{k=1}^\infty \sum_{n=k + 1}^\infty \kappa_n \frac{\gamma^{n-k}}{(n-k)!}
	\frac{u^k}{k!}\\
&= \sum_{k=1}^\infty 
	\bigl(\sum_{n=1}^\infty \kappa_{n+k} \frac{\gamma^n}{n!}\bigr)
	\frac{u^k}{k!}\\
\end{align*}
This shows the cumulants of \(Y^*\) are 
\(\kappa^*_k = \sum_{n=1}^\infty \kappa_{n + k}\gamma^n/n!\)
so we can also compute \(P(Y^* \le z) = G^*(z)\) using the same
technique as we did for \(G\).

If \(G\) is standard normal then all cumulants vanish except \(\kappa_2 = 1\). The only nonzero cumulants of \(G^*\) are \(\kappa_1 = \gamma\)
and \(\kappa_2 = 1\). The corresponding reduced Bell polynomials
are \(B_n = \gamma^n/n!\) and we see the above formula is just
the Taylor series expansion of \(G(z - \gamma)\).
%
%The Edgeworth series expands the exponential in a power series.
%\[\exp\bigl(\sum_{n=0}^\infty \Delta\kappa_n (iu)^n/n!\bigr)
%= \sum_{k=0}^\infty (\sum_{n=0}^\infty \Delta\kappa_n (iu)^n/n!)^k/k!\]
%
\section{K-model}
Kolmogorov's version of the L\'evy-Khintchine theorem\cite{?}
states that if a random variable \(X\) is infinitely divisible
there exists a number \(\gamma\) and a non-decreasing function
\(G\) defined on the real line such that
\[
\log Ee^{iuX} = iu\gamma + \int_{-\infty}^\infty K_u(x)\,dG(x),
\]
where \(K_u(x) = (e^{iux} - 1 - iux)/x^2\). Note the first
cumulant of \(X\) is \(\gamma\) and for \(n\ge 2\),
\(\kappa_n = \int_{-\infty}^\infty x^{n-2}\,dG(x)\). In particular
the variance of \(X\) is 
\(\int_{-\infty}^\infty dG(x) = G(\infty) - G(-\infty)\).

The K-model takes \(\gamma = 0\) and \(G\) of the form
\[
G(x) =
\begin{cases}
a e^{x/\alpha} &x < 0\\
1 - be^{-x/\beta} & x > 0\\
\end{cases}
\]
Note \(G\) jumps by \(1 - a - b\) at the origin. If \(a = b = 0\)
this reduces to a standard normal distribution.

The cumulants are simple to compute: \(\kappa_1 = 0\), \(\kappa_2 = 1\),
and \(\kappa_{n+2} = (a(-\alpha)^n + b\beta^n)n!\) for \(n > 1\).

\section{Remarks}
The Gram-Charlier A series expands the quotients of cumulative
distribution functions \(G/F\) using Hermite polynomials,
but does not have asymptotic convergence, whereas the Edgeworth expansion
involves the quotient of characteristic functions 
\(\hat G/\hat F\) in terms of cumulants and does have asymptotic convergence, ignoring some dainty facts \cite{Petrov}.

\end{document}
%jarrow-rudd
%Arbogast, L. F. A. (1800), Du calcul des derivations (in French), Strasbourg: %Levrault, pp. xxiii+404, Entirely freely available from Google 
%[5] C.A. Charalambides, Enumerative Combinatorics. 
%Chapman & Hall/CRC, 2002. 
%[6] L. Comtet, Advanced Combinatorics, Reidel, Dordrecht, 1974. 
%J.P. Gram, "Ueber die Entwicklung reeller Funktionen in Reihen mittelst der Methode der kleinsten Quadraten" J. Reine Angew. Math. , 94 (1883) pp. 41–73  
%[Ch] C.V.L. Charlier, "Frequency curves of type  in heterograde statistics" Ark. Mat. Astr. Fysik , 9 : 25 (1914) pp. 1–17  
%E. T. Bell, Exponential polynomials, Annals of Mathematics 35 (1934), 258–277

